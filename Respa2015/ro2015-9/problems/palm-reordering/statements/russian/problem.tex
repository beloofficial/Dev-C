\begin{problem}{Пересадка пальм}{F.in}{F.out}{0.5 секунд}{256 мегабайт}

В последнее время пальмы в главном здании Нового Университета (НУ) доставляют много проблем. Контролировать рост деревьев стало непросто даже для самых опытных садовников. Для восстановления внешнего вида здания было решено пересадить некоторые пальмы так, чтобы вся последовательность деревьев шла в неубывающем порядке.

Пересаживать деревья можно в любом порядке, также можно пересаживать некоторые деревья несколько раз. Кроме того, опытные садовники, работающие тут, могут пересадить пальму в любое место между двумя другими, а также в начало или в конец всей последовательности.

К сожалению, пересадка плохо влияет даже на самые лучшие деревья. Поэтому главное условие, которого необходимо придерживаться при работе — количество пальм, которые были хоть раз пересажены, должно быть минимально. 

Опытные садовники также знают, что все пальмы уникальны, и у каждой из них своя стоимость пересадки. Поэтому, прежде чем начать работу, они пытаются определить, какова будет минимальная суммарная стоимость работ, при которой минимальное количество деревьев будет затронуто, а в конце все они будут идти в нужном порядке. А вы можете посчитать это значение?



\InputFile
Первая строка входных данных содержит целое число $N$ ($1 \le N \le 10^5$) --- количество пальм в атриуме университета. 

Во второй строке записано $N$ целых чисел через пробел $H_1,H_2,\dots,H_N$ ($1 \le H_i \le 10^9$) --- высота пальм в том порядке, в котором они растут изначально.

В третьей строке записано $N$ целых чисел через пробел $C_1,C_2,\dots,C_N$ ($1 \le C_i \le 10^9$) --- стоимость пересадки пальм в том же порядке.


\OutputFile
Выведите минимальную суммарную стоимость пересадки пальм, при которой минимальное количество деревьев будет затронуто, а в конце все они будут идти в неубывающем порядке.

\Examples

\begin{example}
\exmp{3
2 1 3
5 6 2
}{5
}%
\end{example}


В первом примере достаточно пересадить только одну пальму --- либо первую (с высотой 2), либо вторую (с высотой 1). Так как стоимость работы с первой пальмой меньше, пересаживается именно она (итоговый ход работ: пальма с высотой 2 пересаживается между пальмами с высотами 1 и 3). 

\Scoring
Данная задача содержит три подзадачи:
\begin{enumerate}
\item $1 \le N \le 20$. Оценивается в $25$ баллов.
\item $1 \le N \le 1000$. Оценивается в $25$ баллов.
\item $1 \le N \le 10^5$. Оценивается в $50$ баллов.
\end{enumerate}

Каждая следующая подзадача оценивается только при прохождении всех предыдущих.

\end{problem}
