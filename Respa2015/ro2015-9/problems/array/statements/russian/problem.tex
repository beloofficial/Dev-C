\begin{problem}{Контейнеры и отсеки}{A.in}{A.out}{0.5 секунд}{256 мегабайт}

Вы главный разработчик в компании грузоперевозок Нурлаш и КО inc. Компании требуется, чтобы вы написали новый функционал для сортирующего робота. Робот контролирует $N$ отсеков, последовательно пронумерованных от 1 до $N$, и может выполнять два типа операций: 

\begin{enumerate}
    \item Добавить контейнер с номером $C$ в каждый отсек с $L$-го по $R$-ый
    \item Убрать последний контейнер из каждого отсека с $L$-го по $R$-ый
\end{enumerate}

Номер контейнера --- целое \textbf{положительное} число не превышающее $10^9$.

Вам даны операции в том порядке в котором их выполнял робот. Требуется определить, для каждого отсека, контейнер с каким номером является последним в нем после выполнения всех операций.

\InputFile
Первая строка входных данных содержит два числа --- $N$, $M$ ($1 \leq N, M \leq 10^5$), количество отсеков и количество операций соответственно.

Далее в $M$ строках содержится по три числа $L$, $R$ и $C$ ($1 \leq L \leq R \leq 10^5$, $0 \leq C \leq 10^9$), описание операций.
Если $C = 0$, то это операция второго типа, иначе --- первого.

Все числа целые и в строках разделены ровно одним пробелом. Также гарантируется, что не будет операций допускающих удаление из пустых отсеков.

\OutputFile
Выведите в единственной строке $N$ чисел, разделенных пробелом.
Первое число --- номер последнего контейнера в первом отсеке, второе - во втором, и т.д.
Если отсек пуст, выведите $0$.

\Examples

\begin{example}
\exmp{5 3
1 5 1
2 4 0
4 5 10
}{1 0 0 10 10 }%
\end{example}


\Scoring
Данная задача содержит две подзадачи:
\begin{enumerate}
\item $1 \leq N, M \leq 1000$. Подзадача оценивается в $40$ баллов.
\item $1 \leq N, M \leq 10^5$. Подзадача оценивается в $60$ баллов.
\end{enumerate}

Каждая следующая подзадача оценивается только при прохождении всех предыдущих.

\end{problem}
